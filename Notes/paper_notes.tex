\documentclass{article}
\usepackage{graphicx}
\usepackage{graphics}
\usepackage{amsthm}
\usepackage{amsmath}
\usepackage{amssymb}
\usepackage{listings}
\usepackage{mathtools}
\usepackage[margin=1in]{geometry}
\usepackage{hyperref}
\usepackage{xcolor}


\newtheorem{theorem}{Theorem}
\newtheorem{definition}{Definition}

\title{Fair Decision Making}
\author{David Yang, Selena She, Amy Feng, Xander Goslin}
\date{Summer 2023}

\begin{document}

\maketitle

\section{Background Reading}
\subsection{\href{https://arxiv.org/abs/1406.2661}{Generative Adversarial Networks}}

\begin{definition}[Generator]
A \textbf{generator} is a neural network that learns to generate realistic samples by transforming random noise into data samples that resemble the training data.
\end{definition}

\begin{definition}[Discriminator]
A \textbf{discriminator} is a neural network that aims to distinguish between real and fake examples.
\end{definition}

More formally, a generative model captures the data distribution and a discriminative model estimates the probability that a sample came from the training data rather than G. "The generative model can be thought of as analogous to a team of counterfeiters, trying to produce fake currency and use it without detection, while the discriminative model is analogous to the police, trying to detect the counterfeit currency."

\begin{definition}
\textbf{Adversarial nets} refer to the specific case where the generative model generates samples by passing random noise through a multilayer perceptron, and the discriminative model is also a multilayer perceptron. 
\end{definition}

\subsection{\href{https://arxiv.org/abs/2006.11239}{Denoising Diffusion Probabilistic Model} / \href{https://arxiv.org/abs/2010.02502}{Denoising Diffusion Implicit Model}}
\end{document}
